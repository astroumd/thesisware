% -*- tex -*-

% andy, olivia? 
% suvi
% upcoming: Zeeve, Thomas, Amy, Corbin, [Pradip=done]


\documentclass[12pt]{article}

\begin{document}

\section{Facilities and Software in a Thesis}

% \bigskip\noindent

\subsection{Introduction}

\label{random}

We are proposing to add a special section {\bf Facilities and
  Software} to a thesis. Like is the case in some current journals,
this would be placed before the {\bf References}. Journals are
increasingly asking for, sometimes even insisting on, this
information, and funders also want to see how facilities, projects,
and data they have paid for are used and contribute to the discipline
(Vishniac and Lintott, 2016\footnote{\tt
  https://iopscience.iop.org/article/10.3847/0004-6256/151/2/21}).
Summarizing this information well would be very helpful for all
stakeholders in the thesis. We maintain the example on
github\footnote{\tt https://github.com/astroumd/thesisware} in case
you want to look for updates or issues.


Below is a template to show how ApJ's \verb+\facilities{}+ and
\verb+\software{}+ could be used in a thesis. Here we show examples in
5 different categories that the student can show of what was used to
complete the thesis. Optionally they can refer (with a URL or page
number) where this was introduced in the thesis.

Before we formally recommend this procedure, we will need to make sure
we such a Section does not violate the Grad School requirements,
otherwise one can also treat it like an Appendix.

%\newpage
% this subsection can be copied into your thesis. The use of \ref{} is optional.
% If you vertical spacing looks too much, try \vspace{-1ex} after item.
% as in normal thesis layout the line spacing is rather large
%
% See https://github.com/astroumd/thesisware for possible updates 
%

\subsection{Facilities and Software used in this Thesis}

\begin{enumerate}
\item
  VLA, see also section \ref{random}
  \newline
      {\it example of data from an existing facility..., reference optional}
      % for some facilities, it is common to use TELESCOPE/INSTRUMENT notation,
      % e.g. SWIFT/BAT , VLT/KMOS,  MAXI/GSC , HERSHELL/SPIRE
      \vspace{-1ex}

\item
  SDDS DR7
  \newline
      {\it example of archival data...}
      % VLA data: 19B-131, 
      % ALMA data: 2016.1.00324.
      \vspace{-1ex}

\item
  Deepthought2, the UMD campus compute server
  \newline
      {\it example of a computing facility...}
      %
      \vspace{-1ex}
      
\item
  cloudy (ASCL:9910.001)   -  or use e.g. \verb+\cite{Ferland2017}+
  \newline
      {\it example of an existing open source code...}
      % feel free to add a \url{} or \footnote{}
      \vspace{-1ex}      

\item
  mycloudy (\verb+https://github.com/astroumd/mycloudy+ - submitted to ASCL)
  \newline
      {\it example of a code developed for the thesis...}
      \vspace{-1ex}      

\end{enumerate}


%\newpage
%% This template was prepared by Pradip Gatkine of the Dept of Astronomy, University of Maryland, College Park, MD with the help of Peter Teuben (same affiliation). 
% The template was last updated in August 2020.  See https://github.com/astroumd/thesisware for possible updates 
% This subsection can be copied into your thesis. The use of \ref{} is optional.



\subsection*{A summary of software/codes used in this work}

\begin{enumerate}
\item
  CASA \cite{emonts2019casa} \vspace{-1ex}

\item
  astropy \cite{robitaille2013astropy} \vspace{-1ex}

\item
  emcee \cite{foreman2013emcee} \vspace{-1ex}

\item
esoreflex \cite{freudling2013automated} \vspace{-1ex}

\item
galfit \cite{peng2003galfit} \vspace{-1ex}

\item
lifelines \cite{cameron_davidson_pilon_2020_3962065} \vspace{-1ex}

\item
ASURV \cite{feigelson1985statistical, isobe1986statistical, isobe1990asurv} \vspace{-1ex}

\item 
FIMMWAVE\footnote{https://www.photond.com} \vspace{-1ex}

\item
Rsoft/BeamPROP\footnote{https://optics.synopsys.com} \vspace{-1ex}

\item
DCAM-CL (Hamamatsu)\footnote{https://dcam-api.com/} \vspace{-1ex}

\end{enumerate}


\end{document}                     
