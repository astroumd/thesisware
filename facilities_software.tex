% -*- tex -*-

% andy, olivia? , kevin
% suvi
% upcoming: Amy, Zeeve, Pradip


\documentclass[12pt]{article}

\begin{document}

\section{Facilities and Software in a Thesis}

% \bigskip\noindent

\subsection*{Introduction}

We are proposing\footnote{Version PJT 13-may-2020}
to encourage our students to make a special section
(an example will be given below) called {\bf Facilities and Software}
to their final thesis. Like is the case in some current journals,
this would be placed before the {\bf References}. Journals are
increasingly asking for, sometimes even insisting on, this
information, and funders also want to see how facilities, projects,
and data they have paid for are used and contribute to the
discipline. Summarizing this information well would be very helpful
for all stakeholders in the thesis.


We need a template to show how ApJ's \verb+\facilities{}+ and
\verb+\software{}+ could be used in a thesis. Here we show examples in
5 different categories that the student can show of what was used to
complete the thesis. Optionally they can refer (with a URL or page
number) where this was introduced in the thesis.


We need to make sure we don't violate some Grad School requirement.

\subsection{Facilities and Software}

\begin{enumerate}
\item
  VLA
  \newline
  {\it example of data from an existing facility...}

\item
  SDDS DR7
  \newline
  {\it example of archival data...}

\item
  Deepthought2
    \newline
  {\it example of a computing facility...}

\item
  cloudy (ASCL:9910.001)
    \newline
  {\it example of an existing open source code...}

\item
  mycloudy (https://github.com/astroumd/mycloudy - submitted to ASCL)
  \newline
    {\it example of a code developed for the thesis...}

\end{enumerate}

 
\end{document}                     
