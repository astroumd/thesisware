% this subsection can be copied into your thesis. The use of \ref{} is optional.
% If you vertical spacing looks too much, try \vspace{-1ex} after item.
% as in normal thesis layout the line spacing is rather large
%
% See https://github.com/astroumd/thesisware for possible updates 
%


% you can make it any part of the thesis, Appendix or (Sub)Section, numbered or not
% as long as it will wind up in the table of contents
\subsection{Facilities and Software used in this Thesis}

\begin{enumerate}
\item
  VLA, see also section \ref{random}
  \newline
      {\it example of data you took from an existing facility..., reference optional}
      % for some facilities, it is common to use TELESCOPE/INSTRUMENT notation,
      % e.g. SWIFT/BAT , VLT/KMOS,  MAXI/GSC , HERSHELL/SPIRE
      % if you or your advisor for example took the data, it belongs here,
      % else it probably belongs in the next catagory of archival data
      \vspace{-1ex}

\item
  SDDS DR7
  \newline
      {\it example of archival data. If you didn't take your VLA data, it should belong in this category}
      % VLA data: 19B-131, 
      % ALMA data: 2016.1.00324.
      \vspace{-1ex}

\item
  Deepthought2, the UMD campus compute server
  \newline
      {\it example of a computing facility...}
      %
      \vspace{-1ex}
      
\item
  cloudy (ASCL:9910.001)   -  or use e.g. \verb+\cite{Ferland2017}+
  \newline
      {\it example of an existing open source code you used...}
      % feel free to add a \url{} or \footnote{}
      \vspace{-1ex}      

\item
  mycloudy (\verb+https://github.com/astroumd/mycloudy+ - submitted to ASCL)
  \newline
      {\it example of a code developed for the thesis, published or not.}
      \vspace{-1ex}      

\end{enumerate}

